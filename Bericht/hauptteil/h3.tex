\section{Genauigkeitsvergleich der beiden Ansätze}
Bei diesem Vergleich geht es darum, die eigenentwickelte
MediaPipe-Pipeline und das zugrunde liegende MediaPipe-Modell zu bewerten,
mit besonderem Blick auf die unterschiedlichen Möglichkeiten und Grenzen
eines monokularen Ansatzes. Zur Einordnung der Ergebnisse wird
die Pose2Sim-Pipeline als Referenz herangezogen, da dort keine eigentliche
3D-Poseschätzung erfolgt, sondern die Positionen der Marker durch das
Multikamera-Setup mittels Triangulierung präziser bestimmt werden können.

Um einen angemessenen Vergleich der beiden Ansätze durchführen zu können,
haben wir uns entschieden, ein identisches OpenSim-Modell zu verwenden.
Verwendet wurde hier das bereits erwähnte arm\_model.osim, befindlich
im GitHub-Repository \cite{github_repository}.
Dafür mussten wir jedoch frühzeitig in die Pose2Sim-Pipeline eingreifen
und die bis dahin erstellte .trc-Datei extrahieren, da im nächsten Schritt
das pipelineeigene Modell erstellt worden wäre.

Um dennoch eine faire Skalierung der Modelle zu ermöglichen, haben wir von
beiden Pipelines eine statische Aufnahme verarbeiten lassen und die daraus
resultierenden .trc-Dateien verwendet, um das jeweilige OpenSim-Modell zu
skalieren.

Anschließend wurde die dynamische Bewegungsdatei im OpenSim-Tool für inverse
Kinematik verarbeitet und die daraus resultierenden Markerfehler extrahiert.
In den verwendeten Videos wird dabei die Bewegung des rechten Arms von einer
nach vorne ausgestreckten Position bis zu einem Ellbogenwinkel von über
90 Grad ausgeführt, wobei die Hand in einer Bewegung zum Gesicht und wieder
zurückgeführt wird. Das Video, welches für MediaPipe verwendet wurde,
zeigt die Person in Seitenansicht, sodass die Bewegung in der sichtbaren x-y-Ebene
stattfindet. Die Tiefeninformation (z-Richtung) wird dabei ausschließlich durch
die 3D-Schätzung des MediaPipe-Modells bestimmt.
Um die Markerfehler in einen Kontext setzen zu können, wurden zudem noch
die von OpenSim erstellten .mot-Dateien extrahiert, die die resultierenden
Gelenkwinkel beinhalten.

Die verwendeten Dateien, die Originalvideos sowie ein Ergebnisvideo,
welches die Eingabevideos gemeinsam mit dem entsprechenden
OpenSim-Ergebnis darstellt, sind im GitHub-Repository \cite{github_repository}
unter \href{https://github.com/PhilippSG/Markerlose-Bewegungserfassung/tree/main/Vergleich}{Vergleich/} wiederzufinden.

Zuerst betrachten wir die aus der inversen Kinematik resultierenden Markerfehler. Diese geben an,
wie nah die Modell-Marker an die Marker der extrahierten Daten angepasst werden
konnten.
\begin{figure}[H]
    \centering
    \includegraphics[width=\textwidth]{img/marker_error_RMS.png}
    \caption{RMS-Fehler der Marker}
    \label{fig:marker_error_RMS}
\end{figure}

In Abbildung~\ref{fig:marker_error_RMS} ist gut zu erkennen, dass die Markerfehler
aus den MediaPipe Daten teils sehr nah an denen von Pose2Sim liegen. Interessant ist,
dass die MediaPipe Markerfehler immer dann ansteigen, wenn die Hand im Video
in die obere Endlage gelangt. Dieses Verhalten lässt sich darauf zurückführen,
dass MediaPipe in dieser Phase durch die stärkere Selbstüberdeckung sowie die
geringe Bewegungsdynamik der Marker eine erhöhte Instabilität der
3D-Rekonstruktion aufweist \cite{Rode2025}. Pose2Sim hingegen liefert hier leicht geringere Markerfehler,
da das Multikamera-Setup eine robustere 3D-Triangulierung ermöglicht und Selbstüberdeckungen
einzelner Marker besser kompensiert werden können.

\begin{figure}[htbp]
    \centering
    \begin{subfigure}{0.49\textwidth}
        \includegraphics[width=\textwidth]{mp_angles.png}
        \caption{Winkel bei MediaPipe}
        \label{fig:mp_angles}
    \end{subfigure}
    \hfill
    \begin{subfigure}{0.49\textwidth}
        \includegraphics[width=\textwidth]{p2s_angles.png}
        \caption{Winkel bei Pose2Sim}
        \label{fig:p2s_angles}
    \end{subfigure}
    
    \begin{subfigure}{0.25\textwidth}
        \centering
        \includegraphics[width=\textwidth]{angles_legend.png}
        \caption{Legende}
        \label{fig:angles_legend}
    \end{subfigure}

    \caption{Winkel Diagramme}
    \label{fig:angle_diagrams}
\end{figure}

Die Abbildung~\ref{fig:angle_diagrams} zeigt die aus der inversen Kinematik
resultierenden Gelenkwinkel beider Ansätze über die Frames. Der vierte
Modellwinkel (r\_elbow\_rotation) wurde dabei bewusst nicht dargestellt, da
dieser bei der ausgeführten Bewegung keine relevante Aussage liefert. Die dort
entstandenen Rotationswerte ergeben sich primär aus den relativen Abständen
und Orientierungen der verwendeten Marker und erlauben daher keinen sinnvollen
Vergleich der beiden Pipelines.

Die Winkelergebnisse zeigen sehr deutlich, dass MediaPipe deutliche Fluktuationen
besonders in den verwendeten Endlagen liefert. Dies lässt vermuten, dass die
Erkennung besser in einer langsamen Bewegung funktioniert, da hier vermutlich die
Bewegungsrichtung und -geschwindigkeit berücksichtigt werden können.

Die größten Abweichungen zeigen sich jedoch, wie zu erwarten, in der 3D Schätzung.
Besonders deutlich wird dies anhand des Winkels r\_shoulder\_side,
der maßgeblich für die Bewegung in z-Richtung ist.
Während MediaPipe hier starke Schwankungen zeigt – insbesondere erneut in der oberen
Endlage – bleibt der entsprechende Winkel bei Pose2Sim über das gesamte Video
hinweg nahezu konstant.
