\chapter{Hauptteil}
\section{Stand der Technik/Modelle für markerlose Bewegungserfassung}
Der aktuelle Stand der Technik basiert überwiegend auf Verfahren der sogenannten Human Pose Estimation (HPE).
Dabei werden anatomisch definierte Gelenkpunkte, also sogenannte Keypoints, direkt aus Bild- oder Videodaten geschätzt.
Moderne Systeme nutzen hierfür überwiegend Deep-Learning-Modelle, insbesondere Convolutional Neural Networks (CNN),
die Gelenkpositionen in Form probabilistischer Heatmaps vorhersagen. Diese Verfahren erreichen in der zweidimensionalen
Pose-Schätzung inzwischen eine sehr hohe Genauigkeit und Robustheit gegenüber variierenden Hintergründen,
Beleuchtungsverhältnissen und Körperformen.

Die dreidimensionale markerlose Bewegungserfassung stellt weiterhin eine größere technische Herausforderung dar.
Ein grundlegendes Problem besteht in der sogenannten Tiefenambiguität, da aus monokularen RGB-Aufnahmen keine
eindeutige Tiefeninformation ableitbar ist. Aktuelle Forschungsansätze begegnen diesem Problem durch die Nutzung
mehrerer synchronisierter Kameras, durch lernbasierte Rekonstruktion dreidimensionaler Posen aus großen annotierten
Datensätzen oder durch zeitliche Modellierung ganzer Bewegungssequenzen. Insbesondere die Kombination räumlicher und
zeitlicher Informationen hat sich als wesentlich für robuste 3D-Rekonstruktionen erwiesen.

Neben rein monokularen Ansätzen kommen zunehmend Multi-View-Systeme zum Einsatz, bei denen mehrere Kameras aus
unterschiedlichen Blickwinkeln verwendet werden. Durch geometrische Triangulation lassen sich dabei dreidimensionale
Gelenkpositionen präziser bestimmen, insbesondere bei Bewegungen mit starker Selbstverdeckung. Diese Systeme erreichen
in kontrollierten Umgebungen eine Genauigkeit, die in bestimmten Anwendungen bereits mit markerbasierten
Motion-Capture-Systemen vergleichbar ist. Allerdings gehen sie mit höherem Hardware-, Kalibrierungs- und Rechenaufwand einher.

Ein weiterer Entwicklungszweig des Stands der Technik sind hybride Verfahren, die visuelle RGB-Daten mit Tiefeninformationen
aus Time-of-Flight- oder Stereo-Kameras kombinieren. Durch die explizite Messung der Entfernung zum Objekt kann die
dreidimensionale Rekonstruktion stabilisiert und die Fehleranfälligkeit bei Abdeckung der Sicht reduziert werden. Solche
Systeme finden insbesondere in der Rehabilitation, der ergonomischen Analyse und der Mensch-Maschine-Interaktion Anwendung.

In den letzten Jahren hat sich zudem der Einsatz fortgeschrittener neuronaler Architekturen wie Graph-Neural-Networks und
Transformer-Modelle etabliert. Diese ermöglichen eine explizite Modellierung der kinematischen Abhängigkeiten zwischen
einzelnen Gelenken sowie der zeitlichen Dynamik von Bewegungen. Aktuelle Forschungsarbeiten untersuchen darüber hinaus den
Einsatz generativer und diffusionsbasierter Modelle, um plausible 3D-Bewegungen auch bei unvollständigen oder verrauschten
Eingangsdaten zu rekonstruieren.

Trotz erheblicher Fortschritte bestehen weiterhin offene Herausforderungen. Dazu zählen die robuste Handhabung von Sichtabdeckung,
die Übertragbarkeit auf neue Personen und Umgebungen sowie die biomechanische Genauigkeit der rekonstruierten Bewegungen.
Insbesondere für Anwendungen, bei denen Gelenkmomente, Muskelkräfte oder Belastungen bestimmt werden sollen, ist die Kopplung
markerloser Bewegungserfassung mit biomechanischen Modellen weiterhin Gegenstand aktueller Forschung.

Zur Analyse und Simulation menschlicher Bewegungen existieren verschiedene etablierte Systeme digitaler Menschmodelle.
Zu den bekanntesten zählen AnyBody, OpenSim, Santos, LifeMOD und Dynamicus.
Ein zentraler Schwerpunkt der Arbeit liegt auf der Erstellung eines reproduzierbaren und praxisnahen Tutorials, das die
vollständige Pipeline der markerlosen Bewegungserfassung beschreibt. Dieses Tutorial umfasst die Installation, Konfiguration
und Anwendung von MediaPipe sowie von Pose2Sim in Kombination mit OpenSim, beginnend bei der Rohdatenerfassung über die
Poseschätzung bis hin zur biomechanischen Auswertung. Durch eine schrittweise und systematische Darstellung soll der Einstieg
für neue Anwender erleichtert und typische Fehlerquellen vermieden werden.

\section{Was ist OpenSim?}
OpenSim ist eine kostenlose Open Source Software Plattform für die Modellierung, Simulierung,
Kontrollierung und Analyse für das Neuro-Muskel-Skelett-System von Menschen, Tieren und Robotern.
Simuliert wird deren Interaktion und Bewegung in der Umgebung, in der sie sich befinden.

\underline{Muskel-Skelett-Modelle}\newline
OpenSim ermöglicht das Erstellen und Verändern von eigenen 3D-Modellen des menschlichen oder
tierischen Bewegungsapparats, inklusive Knochen, Gelenken, Muskeln, Sehnen und anderer Strukturen.
Das graphical user interface kurz GUI erlaubt es, die eigenen Modelle oder bereits vorhandene zu
laden, Visualisierung und ihre Einstellungen im Detail zu bearbeiten.

OpenSim’s Muskelmodelle erfassen die aktiven und passiven generierten Kraft Eigenschaften der
Muskeln, die auf gut getesteten Modellen der Muskel Sehnen Dynamik aus der Literatur basieren.

Die Softwareplattform kommt mit einer großen Anzahl an Muskel Skelett Modellen, die beim
Herunterladen bereits zur Verfügung gestellt sind, bei denen es sich um Modelle des menschlichen
Oberkörpers und der unteren Extremitäten handelt. Zusätzlich zu den Modellen, die in
Forschungsqualität angeboten werden, gibt es weitere teilweise stark vereinfachte Modelle für
die Erstellung von Prototypen und zum Lehren. Außerdem bietet OpenSim die Möglichkeit von Benutzern
selbst erstellte Modelle aus einer Online Bibliothek zu laden.

Zum Analysieren und Simulieren von Modellen und Bewegungen gibt es Tools, um Markerdaten,
Gelenkkinematiken und externe Kräfte zu importieren und ebenfalls zu visualisieren. Einige Benutzer
haben Toolboxen erstellt und geteilt, die mit verschiedenen gängigen Bewegungserfassungssystemen
kompatibel sind.

\underline{Tools}\newline
Besonders von Bedeutung für dieses Projekt sind zum einen das Scale-Tool und zum anderen das Inverse
Kinematics (IK)-Tool.

Das Scale-Tool dient hierbei zur Anpassung der Modellgeometrie an die individuellen Eigenschaften der Körpermaße
einer Versuchsperson. Die Skalierung basiert auf einer Kombination aus den gemessenen Abständen zwischen den
experimentell erfassten x-y-z-Markerpositionen sowie einigen optional manuell vorgegebenen Skalierungsfaktoren.
Die Markerpositionen werden in der Regel mit den in der Einleitung genannten Motion-Capture-Systemen aufgezeichnet.
Das unskalierte Modell enthält korrespondierende virtuelle Marker, die an denselben anatomischen Landmarken
wie die experimentellen Marker positioniert sind.
Während des Skalierungsprozesses werden die Abmessungen der einzelnen Körpersegmente so angepasst, dass
die Abstände zwischen den virtuellen Markern (Modell Marker) und den gemessenen Abständen der experimentellen Markern entsprechen.
Ergänzend oder alternativ dazu können zu der markerbasierten Skalierung auch manuell definierte Skalierungsfaktoren
für einzelne Körpersegmente verwendet werden, die beispielsweise aus unabhängigen anthropometrischen Messungen stammen.
Nach Abschluss der Skalierung ermöglicht das Scale-Tool zudem, ausgewählte oder alle virtuellen Marker gezielt
zu verschieben, sodass sie exakt mit den experimentellen Markerpositionen übereinstimmen.
Dadurch erricht sich eine verbesserte Übereinstimmung zwischen dem gewählten Modell und den Versuchsdaten aus der Bewegungserkennung, 
womit eine Grundlage für nachfolgende Analysen geschaffen wird.

Die Inverse Kinematic (IK) Tool berechnet für jeden Zeitschritt (frame) der einer Bewegung die generalisierten Koordinaten
des Modells, sodass dessen Pose bestmöglich mit den experimentell erfassten Marker- und gegebenfalls Koordinatendaten übereinstimmt.
Die beste Übereinstimmung wird mathematisch als ein gewichtetes Least Square Problem (kleinstes Fehlerquadrat) formuliert, bei dem sowohl Marker als auch
Koordinatenfehler minimiert werden.
(Bild der Rechnung)
Die Markerfehler die dabei entstehen sind definiert als der euklidische Abstand zwischen der Position eines experimentell bestimmten Markers
und der entsprechenden Markerposition am virtuellen Modell, nachdem dieses durch die berechnete generalisierte Koordinaten positioniert wurde.
Wie bereits erwähnt, ist jedem Marker ein Gewicht zugeordnet, das bestimmt, wie stark dessen Abweichung in die Gesamtfehlerfunktion eingeht.
Analog dazu werden auch Koordinatenfehler definiert, die die Differenz zwischen einem experimentell vorgegebenen Koordinatenwert, beispielsweise 
einem Gelenkwinkel, und dem durch die IK berechneten Wert beschreiben.
Außerdem können die Experimentellen Koordinaten, die wiederum aus den Motion Capture Systemen, speziellen Auswertealgorithmen oder zusätzlichen
Messgeräten wie Goniometern stammen, alternativ auch mit festen Sollwerten vorgegeben werden.
Feste Sollwerte sind insbesondere dann sinnvoll, wenn bestimmte Gelenkwinkel konstant bleiben sollen.
Bei der IK-Berechnung wird zwischen vorgeschriebenen und nicht vorgeschriebenen Koordinaten unterschieden.
Vorgeschriebene Koordinaten besitzen einen bekannten Verlauf und werden während der IK nicht berechnet, sondern direkt auf ihre
vorgegebenen Werte gesetzt (z.B. fixierte Gelenke).
Nicht vorgeschriebene Koordinaten werden hingegen durch die IK bestimmt und gehen in das Optimierungsproblem (Least Square) ein.
Wie bei den Markern können auch den Koordinaten unterschiedliche Gewichte zugewiesen werden, die deren Einfluss auf dei Koordinatenfehler
bestimmen.

Die Optimierung ist sensitiv gegenüber der Wahl der Einheiten.
In der verwendetetn Modellumgebung werden Meter für Längen und Radiant für Winkel verwendet.
Dies ist besonders beim Vergleicht mit anderen IK-Verfahren relevant, die Winkel in Grad verarbeiten.
Um vergleichbare Ergebnisse zu erhalten, müssen die Koordinatengewichte entsprechend an die verwendetetn Winkeleinheiten angepasst werden


Mit der statischen Optimierung kann das Muskel-Redundanzproblem gelöst werden, basierend auf
etablierten Verfahren aus der wissenschaftlichen Literatur.

Mit Hilfe des Computed Muscle Control (CMC)-Tool lassen sich muskelgetriebene Vorwärtssimulationen
erzeugen. Diese Methode wurde erfolgreich für verschiedene Bewegungen wie das Gehen, Laufen,
Radfahren, Springen und die Analyse pathologischer Gangmuster eingesetzt.

OpenSim ermöglicht außerdem eine detaillierte Analyse („Probing“) von Modellen und Simulationen.
Dabei können Größen wie Gelenkwinkel, Muskelkräfte, Muskelhebelarme, Muskelarbeit oder die Bewegung
des Körperschwerpunkts untersucht und grafisch dargestellt werden.

Zur Visualisierung bietet OpenSim eine grafische Benutzeroberfläche, in der nahezu alle
Modellkomponenten angezeigt werden können. Externe Geometrien (z. B. STL- oder OBJ-Dateien)
lassen sich importieren und mit integrierten Bild- und Video-Tools können anschauliche Darstellungen
für Präsentationen und Publikationen erstellt werden.
