\chapter{Hauptteil}
\section{Ablauf bei der Verwendung von MediaPipe}
Im folgenden wird unserer Workflow beschrieben, den wir verwendet haben um mit MediaPipe zum Ziel zu kommen. 
\subsection{Importierung der Bibliotheken}
Es werden folgende Versionen der Module für die Verwendung mit Python verwendet:
\begin{itemize}
    \item csv (kommt mit python) 3.12.12 \newline
    Standard-Modul zum Schreiben von Textdateien im "Comma Separated Values" - Format.
    \item math (kommt mit python) 3.12.12 \newline
    Standard-Mathematik-Bibliothek von Python.
    \item cv2 4.12.0.88 (opencv-python) \newline
    Bibliothek für Computer Vision. Ist Notwendig zum lesen der Video Dateien und weiteren Bearbeitung. 
    \item MediaPipe 0.10.21 \newline
    Machine-Learning-Bibliotheken von Google. Wichtig für die Pose Schätzung.
    \item pandas 2.3.3 \newline
    Modul für die Datenanalyse und Tabellenkalkulation. Wichtig für die Umwandlung von CSV in TRC Datei.
    \item numpy 1.26.4 \newline
    Standardbibliothek für mathematische Berechnungen mit Vektoren und Matrizen. Wichtig für die Skalierung.
\end{itemize}

\subsection{Anzeigen der Videos}

\subsection{Erstellen einer CSV mit MediaPipe Daten aus einem brauchbaren Video}
\subsection{Skalierung der Daten zur Verwendung in OpenSim}
\subsection{Erstellen einer Tracer(.trc) Datei aus den Daten in der CSV Datei}
\subsection{Verwendung der Tracer Datei um die virtuellen Marker als experimentelle Marker in OpenSim anzeigen zu lassen}
\subsection{Anwendung auf ein Modell in OpenSim}
