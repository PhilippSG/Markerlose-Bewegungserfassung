\section{Ablauf bei der Verwendung von Pose2Sim}
Im folgenden wird der Workflow bei der Verwendung von Pose2Sim beschrieben.

\subsection{Systemvoraussetzungen}
Für die Verwendung von Pose2Sim werden folgende Systemvoraussetzungen benötigt:
\begin{itemize}
    \item \textbf{Hardware:} Ein Computer mit ausreichend Speicher und Rechenleistung.
    \item \textbf{Software:} Python, Miniconda oder Anaconda.
    \item \textbf{Abhängigkeiten:} Verschiedene Python-Module und Bibliotheken.
\end{itemize}

\subsection{Installation von Pose2Sim}
Die Installation von Pose2Sim erfolgt über Miniconda oder Anaconda. Nach der Installation sollte die Installation getestet werden. Typische Installationsprobleme und deren Lösungen werden in der offiziellen Dokumentation beschrieben.

\subsection{Erstellen eines OpenSim Modells mit Pose2Sim}
\subsubsection{Ordnerstruktur}
Bei der Verwendung von Pose2Sim ist eine spezifische Ordnerstruktur erforderlich.

\subsubsection{Kamera Setup}
Das Kamera Setup ist ein wichtiger Bestandteil des Workflows. Hier können verschiedene Fehlerquellen auftreten.

\subsubsection{Kalibrierung der Kameras}
Die Kameras müssen sowohl intrinsisch als auch extrinsisch kalibriert werden.

\subsubsection{Konfigurationsdatei}
Die Konfigurationsdatei definiert wichtige Parameter des Pose2Sim Workflows. Der Aufbau und der Zweck dieser Datei werden hier erläutert.

\subsubsection{Software und Plugins}
Für die synchronisierte Aufnahme werden spezielle Software-Tools und Plugins benötigt.

\subsubsection{2D Pose Erkennung}
Die 2D Pose Erkennung ist der erste Schritt im Workflow.

\subsubsection{3D Modell}
Aus den 2D Daten wird ein 3D Modell erstellt.

\subsubsection{TRC Datei Erstellung}
Die 3D Daten werden in eine TRC Datei konvertiert.

\subsubsection{Integration in OpenSim}
Die TRC Datei kann nun in OpenSim integriert werden. Typische Fehlerquellen werden hier aufgezeigt.

\paragraph{Synchronisieren mehrerer Modelle in OpenSim}
Um mehrere Modelle und ihre Bewegungen synchron abzuspielen:
\begin{enumerate}
    \item Modelle und zugehörige Motion-Dateien in OpenSim laden.
    \item Im linken Reiter die Option \enquote{coordinates} aller relevanten Modelle gleichzeitig markieren (Strg + Linksklick).
    \item Rechtsklick auf eine der markierten Optionen und \enquote{Sync. Motions} wählen.
\end{enumerate}

\subsection{Visualisierung und Qualitätscheck}
Vor der Verwendung in OpenSim sollte eine Visualisierung und ein Qualitätscheck durchgeführt werden. Hierbei können Hardware Limitierungen zutage treten.

\subsection{Zusätzliche Skripte}
Es stehen Skripte zur Verfügung für:
\begin{itemize}
    \item Winkelkorrektur
    \item Steuerung
\end{itemize}

\subsection{Optimiertes Pose2Sim Skript}
Eine optimierte Version des Pose2Sim Skripts steht bereit. Schritte zur Nutzung:
\begin{enumerate}
    \item \href{https://github.com/PhilippSG/Markerlose-Bewegungserfassung/tree/main/Pose2Sim}{Readme im Repository lesen} (Pose2Sim/readme.md).
    \item Anleitung aus dem Readme befolgen und Skript ausführen.
\end{enumerate}

\subsection{Einfluss unterschiedlicher Aufnahmeszenarien}
Die Robustheit und Ergebnisqualität werden durch verschiedene Aufnahmeszenarien beeinflusst:
\begin{itemize}
    \item Einzelpersonen-Aufnahme vs. Multi-Personen-Aufnahmen
    \item Helle vs. dunkle Kleidung
    \item Enge vs. weite Kleidung
    \item Gute vs. schlechte Belichtung
    \item Große vs. kleinere Personen
\end{itemize}

\subsection{Beobachtungen und Grenzen}
\paragraph{Positive Beobachtungen}
\begin{itemize}
    \item Korrekte Abbildung von Größenunterschieden und Kopfformen.
    \item Erkennung von visuellen Illusionen, wenn Personen im Hintergrund Gesten ausführen, die vordergründig wirken.
\end{itemize}

\paragraph{Fehlgeschlagene Experimente}
\begin{itemize}
    \item Experimente im Dunkeln mit dunkler Kleidung (Teleportation, schnelle oder langsame Kniebeugen) schlagen fehl; ausreichende Beleuchtung ist erforderlich.
    \item Handheben mit heller Kleidung bei guter Beleuchtung funktioniert, erzeugt aber teils Rauschen.
    \item Rollstuhl- oder Stuhlbewegungen führen zu ruckeligen Simulationen (Hüfte und Hände springen).
\end{itemize}

\subsection{Fazit}
Abschließend werden die Grenzen und die Anwendbarkeit von Pose2Sim bewertet. Weitere Informationen finden sich auf der GitHub-Seite des Projekts.

