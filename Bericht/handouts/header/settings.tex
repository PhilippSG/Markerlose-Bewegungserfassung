% !TEX root = ../mediapipe.tex
\documentclass[11pt,
    a4paper,
    ngerman,
    numbers=noenddot, % not 1.1. but 1.1 
    oneside,
    bibliography=totocnumbered,
    listof=totoc,
    parskip=half,
]{scrreprt}

%\usepackage{parskip}
%\usepackage{geometry}

% Fonts
\usepackage{helvet}
\usepackage{lmodern}
\renewcommand{\familydefault}{\sfdefault}
% Color
\usepackage{xcolor}

% Für Grafiken
\usepackage{graphicx}
\usepackage{subcaption}

% Search path for images  
\graphicspath{{../img}}


\usepackage{setspace}   
\usepackage{tabularx}  

% Für die Sprache und Schrift )
\usepackage[ngerman]{babel} 
\usepackage[T1]{fontenc}
\usepackage{lmodern}
\usepackage{microtype}

%Code in LaTeX darstellen
\usepackage{listings}
\usepackage{xcolor}

%Hyperlinks
\usepackage[hidelinks]{hyperref}
\usepackage{xurl}
\usepackage{bookmark}

% Anführungszeichen
\usepackage{csquotes}

% --- Farben definieren (VS Code Light Theme) ---
\definecolor{vsBlue}{RGB}{0, 0, 255}          % Keywords (def, import, return)
\definecolor{vsGreen}{RGB}{0, 128, 0}         % Kommentare
\definecolor{vsRed}{RGB}{163, 21, 21}         % Strings ("...")
\definecolor{vsPurple}{RGB}{175, 0, 219}      % Built-ins oder Zahlen (optional)
\definecolor{vsGray}{RGB}{128, 128, 128}      % Zeilennummern
\definecolor{vsBack}{RGB}{250, 250, 250}      % Leichter Grauschleier als Hintergrund

% --- Style Definition ---
\lstdefinestyle{vscode}{
    backgroundcolor=\color{vsBack},   
    commentstyle=\color{vsGreen},
    keywordstyle=\color{vsBlue}\bfseries, % Keywords fett und blau
    numberstyle=\tiny\color{vsGray},
    stringstyle=\color{vsRed},
    basicstyle=\ttfamily\footnotesize, % Schriftart: Typewriter, etwas kleiner
    breakatwhitespace=false,         
    breaklines=true,                 % Zeilenumbruch aktivieren
    captionpos=b,                    % Beschriftung unten
    keepspaces=true,                 
    numbers=left,                    % Zeilennummern links
    numbersep=5pt,                  
    showspaces=false,                
    showstringspaces=false,
    showtabs=false,                  
    tabsize=4,                       % Tabulator-Breite
    frame=single,                    % Rahmen um den Code
    rulecolor=\color{lightgray},     % Rahmenfarbe
    language=Python,                 % Sprache festlegen
    morekeywords={self, as},         % Fehlende Keywords manuell ergänzen
}

% --- Style aktivieren ---
\lstset{style=vscode}