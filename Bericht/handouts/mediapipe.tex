%Einfügen der Header-Dateien
% !TEX root = ../main.tex
\documentclass[11pt,
    a4paper,
    american,
    numbers=noenddot, % not 1.1. but 1.1 
    oneside,
    bibliography=totocnumbered,
    listof=totoc,
    parskip=half,
]{scrreprt}

%\usepackage{parskip}
%\usepackage{geometry}

% Fonts
\usepackage{helvet}
\usepackage{lmodern}
\renewcommand{\familydefault}{\sfdefault}
% Color
\usepackage{xcolor}

% Graphics
\usepackage{graphicx}
% Search path for images  
\graphicspath{{img/}}


\usepackage{setspace}   
\usepackage{tabularx}  

% Für die Sprache und Schrift )
\usepackage[american]{babel} 
\usepackage[T1]{fontenc}
\usepackage{lmodern}
\usepackage{microtype}

% Start des Dokuments
\begin{document}
\setcounter{chapter}{0}
\renewcommand{\thesection}{\arabic{section}} % Sections nur arabisch nummerieren
\newpage
\begin{center}
    {\LARGE \textbf{Handout: MediaPipe-Pipeline}}\\[0.5cm]
\end{center}

\section{Schritt-für-Schritt Anleitung}
In diesem Handout wird die korrekte Verwendung der erstellten MediaPipe-Pipeline an einem
Beispiel erläutert. Die verwendeten Videos finden sich unter dem Ordner
Markerlose-Bewegungserfassung\//Videos\//MP\_2d3d\_comp. In dem Beispiel wird ein einfaches
Arm-Modell verwendet, welches auf 3 Markern basiert (Schulter, Ellbogen und Handgelenk). Das
verwendete OpenSim Modell ist das im GitHub-Repository befindliche
Markerlose-Bewegungserfassung\//OpenSim\//models\//arm\_modell.osim.

Detailliertere Erklärungen zu den einzelnen Schritten finden Sie im Kapitel 2.4 des Hauptberichts.

\subsection{Repository anlegen}
Folgenden Befehl an gewünschter Stelle im Dateisystem ausführen:
\begin{verbatim}
    git clone https://github.com/PhilippSG/Markerlose-Bewegungserfassung.git
\end{verbatim}

\subsection{Umgebung einrichten}
Python Version 3.12.12 verwenden und Pakete mit den in der Sektion Softwareversionen und Module des Berichts
dargestellten Versionen installieren.
\subsection{Videos erstellen}
Mindestens 2 Videos werden benötigt: statische (z.B. T-Pose, ca. 3-5 Sekunden halten) und
dynamische Aufnahme mit der gewünschten Bewegung. Genauere Anforderungen an die Videos sind im Bericht
in der Sektion Videos beschrieben.

Die Videos müssen unter dem Ordner Markerlose-Bewegungserfassung\//Videos abgelegt oder die Pfade
im Code entsprechend angepasst werden. 
\subsection{Videos laden und anzeigen}
Führen Sie das Jupyter Notebook \texttt{Workflow.ipynb} bis zur Zelle \texttt{Execute} aus.
In der Zelle darunter finden sich die Variablen \texttt{static\_video\_name} und
\texttt{dyn\_video\_name}. Setzen Sie diese auf die entsprechenden Videonamen (nur Dateiname, ohne Dateiendung).
Die Dateiendung sollte mp4 sein, ansonsten muss diese in den Variablen \texttt{static\_video\_path} und
\texttt{dyn\_video\_path} entsprechend angepasst werden.
Führen Sie nun diese und die darauffolgende Zelle aus, um die Videos über OpenCV zu laden und
anzuzeigen. Sollten die Videos nicht in der richtigen Ausrichtung abgespielt werden, so muss die
Variable \texttt{rotate\_video\_by\_degrees} entsprechend angepasst werden. Rotiert wird dabei
im Uhrzeigersinn.

\subsection{Erstellen der Landmarks}
In der nächsten Zelle findet sich die \texttt{landmark\_map}. In den Kommentaren finden
sich bereits alle von MediaPipe BlazePose verfügbaren Marker. Hier müssen Sie
die gewünschten Marker mit den Namen der Marker des OpenSim Modells anpassen.
Die Marker des \texttt{arm\_modell.osim} heißen
\texttt{r\_acromion, r\_humerus\_epicondyle und r\_radius\_styloid} und entsprechen in der Reihenfolge
den Markern rechte Schulter, Ellbogen und Handgelenk.

\subsection{Ausführen der Pose Estimation}
In diesem Schritt findet die eigentliche Pose Estimation von MediaPipe statt.
Vor dem jeweiligen Aufruf der \texttt{pose\_estimation} Funktion findet sich die
Variable \texttt{rotate\_around\_y\_axis}. Diese sorgt, wie der Name vermuten lässt,
für eine Rotation der Daten um die y-Achse. Damit kann die Ausrichtung auf die Ausrichtung
des OpenSim Modells angepasst werden, um sicherzustellen, dass die Marker auf die des Modells
angepasst sind. Rotiert wird dabei gegen den Uhrzeigersinn.

Die Standardausrichtung des OpenSim Modells entspricht einer Person,
die im Video nach rechts schaut. Die Beispielvideos müssen also wie in Abbildung~\ref{fig:example_orientations}
dargestellt rotiert werden.
\begin{figure}[htbp]
    \centering
    \begin{subfigure}{0.44\textwidth}
        \includegraphics[width=\textwidth]{handouts/example_facing_cam.png}
        \caption{\texttt{rotate\_around\_y\_axis=90}}
        \label{fig:rotate90}
    \end{subfigure}
    \hspace{0.5cm}
    \begin{subfigure}{0.35\textwidth}
        \includegraphics[width=\textwidth]{handouts/example_facing_right.png}
        \caption{\texttt{rotate\_around\_y\_axis=0}}
        \label{fig:rotate0}
    \end{subfigure}
    \caption{Beispiel Orientierungen mit notwendiger Rotation}
    \label{fig:example_orientations}
\end{figure}
Das Ausführen dieser Zelle kann je nach Modellkomplexität und Dauer der
Videos einige Minuten in Anspruch nehmen.
\newpage
\subsection{Skalierung}
Anschließend müssen die Daten skaliert werden, um bessere Ergebnisse zu erreichen.
Dafür müssen die Körperteile in Gruppen in dem Array \texttt{limb\_groups} übergeben werden.

Wenn nach einer gemessenen Länge skaliert werden soll, kann diese in der Variablen
\texttt{reference\_length\_for\_scaling} und das Körperteil in der Variablen \newline
\texttt{reference\_limb\_for\_scaling} übergeben werden.

Zudem findet bei der Skalierung auch die Ursprungsverschiebung statt. Dafür muss ein Marker
gewählt werden, nach dem sich diese Verschiebung orientiert. Tragen Sie den Namen dieses Markers
in der Variable \texttt{anchor\_marker} ein und übergeben Sie die Position dieses Markers im
OpenSim Modells in der Variable \texttt{anchor\_vector}.
Schalten Sie die Ausgabe der Skalierungsdiagramme über den Parameter \texttt{plots}
der \texttt{scale} Funktion ein, um die Skalierung kontrollieren zu können.

\subsection{Formatierung zu .trc}
Führen Sie zuletzt die letzte Zelle aus, um die bisher verwendeten .csv Dateien in .trc
Dateien zu konvertieren, die dann in OpenSim geladen werden können.
\end{document}
