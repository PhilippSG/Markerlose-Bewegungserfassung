% Auskommentierter Teil wurde im Hauptteil umgeschrieben und verwendet

\chapter{Einleitung}
\section{Problemstellung und Motivation}
Die Analyse und Rekonstruktion menschlicher Bewegungen spielt in vielen Bereichen eine wichtige Rolle, darunter
Sportwissenschaft, Rehabilitation, Animation und Robotik. Traditionell werden hierfür markerbasierte
Bewegungserfassungssysteme eingesetzt, die jedoch mit hohen Kosten, aufwändigen Vorbereitungen und eingeschränkter
Flexibilität verbunden sind. In den letzten Jahren haben sich markerlose Methoden zur Pose-Erfassung entwickelt, die
diese Herausforderungen adressieren und eine einfachere sowie kostengünstigere Alternative bieten.
Die markerlose Erfassung menschlicher Posen gewinnt zunehmend an Bedeutung, da sie eine flexible und
alltagstaugliche Alternative zu markerbasierten Bewegungserfassungssystemen darstellt und vielfältige Anwendungen bietet.

Markerlose Bewegungserfassungssysteme liefern in der Regel kinematische Informationen über die räumliche Position und Bewegung
einzelner Körpersegmente oder Gelenkpunkte.
Diese Verfahren ermöglichen eine effiziente Rekonstruktion der menschlichen Bewegungen, sind jedoch auf die reine Beschreibung der Bewegung beschränkt.
Aussagen über Biomeschanische Größen wie Gelenkwinkel, Gelenkmomente, Muskelkräfte oder innere Belastungen sind mit solchen Erfassungssystem alleine nicht möglich,
wobei für dieses Projekt die Gelenkwinkel im Fokus stehen.
Für weitergehende Analysen von menschlichen Bewegungen ist daher eine zusätzliche biomechanische Modellierung erforderlich,
welche den menschlichen Bewegungsapparat unter Berücksichtigung seiner anatomischen und physikalischen Zusammenhänge abbildet.
Biomechanische Simulationsumgebungen wie z.B. OpenSim, die in diesem Projekt verwendet werden, ermöglichen es,
aus den kinematischen Eingangsdaten (Landmarks) dynamische Kenngrößen abzuleiten und Bewegungen hinsichtlich ihrer mechanischen Belastung zu untersuchen.
Die Kombination markerloser Bewegungserfassungssysteme mit biomechanischen Modellen erlaubt somit ganzheitliche Analysen menschlicher Bewegungen,
bei der die Vorteile einer flexiblen und kostengünstigen Datenerfassung mit der Aussagekraft physikalisch fundierter Simulationen verbunden werden kann.

%In dieser Projektarbeit werden zwei Ansätze zur markerlosen Pose-Erfassung betrachtet und miteinander verglichen.
%Untersucht werden zum einen MediaPipe als kamerabasierter Ansatz zur 2D- und 3D-Poseschätzung sowie zum anderen eine
%OpenSim-basierte Pipeline unter Verwendung von (RTM Pose und) Pose2Sim zur Rekonstruktion biomechanischer Modelle.


%Ein weiterer Schwerpunkt dieser Arbeit liegt auf einem Genauigkeitsvergleich der Verfahren, wobei Pose2Sim als Referenz für die
%Bewertung herangezogen wird. Ziel ist es, die Genauigkeit von MediaPipe im Vergleich zur Pose2Sim-Pipeline zu bewerten und deren
%Eignung für biomechanische Anwendungen einzuschätzen. Darüber hinaus wird die Robustheit der Pose2Sim-Pipeline unter
%unterschiedlichen Aufnahmeszenarien untersucht. Dabei wird analysiert, inwiefern Faktoren wie Kleidung, Beleuchtungsbedingungen, etc.
%die Genauigkeit der Poseschätzung und der rekonstruierten Modelle beeinflussen.

% Abschließend werden die jeweiligen Vor- und Nachteile der untersuchten Ansätze verglichen und geeignete Einsatzgebiete identifiziert.
% Aufbauend auf den Ergebnissen dieser Arbeit wird zudem eine potentielle weiterführende Aufgabe formuliert, die es einer nachfolgenden
% Projektgruppe ermöglicht, die vorgestellten Ansätze weiterzuentwickeln, zu validieren oder auf erweiterte Datensätze und Anwendungsszenarien anzuwenden.
