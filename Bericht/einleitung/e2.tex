\section{Was ist markerlose Bewegungserfassung?}
Ziel der Bewegungserfassung ist die realitätsnahe Abbildung des menschlichen Bewegungsapparats,
um (menschliche) Bewegungen digital zu erfassen und auszuwerten. Die Bewegungserfassung kann
dabei entweder markerbasiert oder markerlos erfolgen.

Bei markerbasierten Verfahren werden retroreflektierende Marker an einer Person angebracht,
von mehreren Kameras erfasst und anschließend dreidimensional rekonstruiert.

Markerlose Bewegungserfassung verzichtet vollständig auf physische Markierungen am Körper.
Stattdessen erfolgt die Bewegungsrekonstruktion ausschließlich bildbasiert mithilfe kalibrierter Kameras.
Die Position wird im Anschluss mittels Triangulation dreidimensional rekonstruiert.
(https://www.mpg.de/417219/forschungsSchwerpunkt1 - Markerfreies Motion Capture:
Neue Wege zur Analyse menschlicher Bewegungen)

Markerlose Bewegungserfassung ermöglicht eine größere Flexibilität in der Anwendung,
ist weniger zeit- und kostenintensiv und damit für den Alltagsgebrauch besser geeignet.
(https://www.mpg.de/417219/forschungsSchwerpunkt1 - Markerfreies Motion Capture:
Neue Wege zur Analyse menschlicher Bewegungen) Im Vergleich zu markerbasierten Systemen ist
sie jedoch mit einer geringeren Genauigkeit sowie einer höheren Latenz verbunden.
(Homo Sapiens Digitalis – Virtuelle Ergonomie und digitale Menschmodelle, S. 275).

Die Einsatzgebiete der markerlosen Bewegungserfassung umfassen sowohl die Sportwissenschaften
als auch die Medizin, insbesondere in den Bereichen Rehabilitation, Orthopädie und Ganganalyse
(Webbasierte Anwendungen Virtueller Techniken, S. 86). Darüber hinaus findet sie Anwendung in der
Film- und Spieleentwicklung, beispielsweise in der Animation virtueller Charaktere, sowie in der
Robotik und der Mensch-Maschine-Interaktion.\newline
(https://www.mpg.de/417219/forschungsSchwerpunkt1 -
Markerfreies Motion Capture: Neue Wege zur Analyse menschlicher Bewegungen)
