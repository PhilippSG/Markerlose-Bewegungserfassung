\section{Was ist markerlose Bewegungserfassung?}
Ziel der Bewegungserfassung ist die realitätsnahe Abbildung des menschlichen Bewegungsapparats, um (menschliche) Bewegungen digital zu erfassen und auszuwerten. Die Bewegungserfassung kann dabei entweder markerbasiert oder markerlos erfolgen.

Bei markerbasierten Verfahren werden retroreflektierende Marker 
an definierten anatomischen Landmarken einer Person angebracht und von mehreren Kameras erfasst.
\cite{mpg}
Alternativ können inertiale Messeinheiten (IMUs) eingesetzt werden,
die Bewegungsdaten ohne optische Kamerasysteme erfassen,
indem sie mehrere integrierte Sensoren kombinieren. 
Die gewonnenen Messdaten werden in beiden Fällen mithilfe spezieller Software verarbeitet und auf digitale Modelle übertragen.
\cite{mimic_suits}

Markerlose Bewegungserfassung verzichtet vollständig auf 
physische Markierungen am Körper und erfolgt ausschließlich bildbasiert. 
Bewegungen werden mithilfe von Kameras aufgezeichnet und softwaregestützt ausgewertet, 
wobei menschliche Körpersegmente und Gelenkpunkte durch Bildverarbeitungsverfahren identifiziert werden. 
Die ermittelten zweidimensionalen Positionsdaten werden im Anschluss mittels Triangulation dreidimensional rekonstruiert. 
\cite{mpg}

Markerlose Bewegungserfassung ermöglicht eine größere Flexibilität in der Anwendung, 
ist weniger zeit- und kostenintensiv und damit für den Alltagsgebrauch besser geeignet. \cite{mpg} 
Im Vergleich zu markerbasierten Systemen ist sie jedoch mit einer geringeren Genauigkeit sowie einer höheren Latenz verbunden.
\cite[S.~275]{homo_sapiens_digitalis}

Die Einsatzgebiete der markerlosen Bewegungserfassung umfassen sowohl die Sportwissenschaften als auch die Medizin, 
insbesondere in den Bereichen Rehabilitation, Orthopädie und Ganganalyse.\cite[S.~86]{web_vr_ar} 
Darüber hinaus findet sie Anwendung in der Film- und Spieleentwicklung, beispielsweise in der Animation virtueller Charaktere, 
sowie in der Robotik und der Mensch-Maschine-Interaktion. \cite{mpg}

% Quelle und Bild nur in Google Docs ("IP_Einleitung")
% Bilddatei marker_vs_markerlos.png fehlt noch im Projektordner; Quelle im Quellenverzeichnis fehlt noch!

\begin{figure}[H]
    \centering
    \includegraphics[width=0.4\textwidth]{Abb_1_2_marker_vs_markerlos.png}
    \caption{Vergleich der markerbasierten (oben) und markerlosen (unten) Bewegungserfassung von der Datenerfassung bis zur dreidimensionalen Rekonstruktion. (Quelle: \cite{marker_vs_markerless_img})}
    \label{fig:markerbasiert_vs_markerlos}
\end{figure}
