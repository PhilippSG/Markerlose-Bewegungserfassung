\section{Was ist OpenSim?}
OpenSim ist eine kostenlose Open Source Software Plattform für die Modellierung, Simulierung,
Kontrollierung und Analyse für das Neuro-Muskel-Skelett-System von Menschen, Tieren und Robotern.
Simuliert wird deren Interaktion und Bewegung in der Umgebung, in der sie sich befinden.

\underline{Muskel-Skelett-Modelle}\newline
OpenSim ermöglicht das Erstellen und Verändern von eigenen 3D-Modellen des menschlichen oder
tierischen Bewegungsapparats, inklusive Knochen, Gelenken, Muskeln, Sehnen und anderer Strukturen.
Das graphical user interface kurz GUI erlaubt es, die eigenen Modelle oder bereits vorhandene zu
laden, Visualisierung und ihre Einstellungen im Detail zu bearbeiten.

OpenSim’s Muskelmodelle erfassen die aktiven und passiven generierten Kraft Eigenschaften der
Muskeln, die auf gut getesteten Modellen der Muskel Sehnen Dynamik aus der Literatur basieren.

Die Softwareplattform kommt mit einer großen Anzahl an Muskels Skelett Modellen, die beim
Herunterladen bereits zur Verfügung gestellt sind, bei denen es sich um Modelle des menschlichen
Oberkörpers und der unteren Extremitäten handelt. Zusätzlich zu den Modellen, die in
Forschungsqualität angeboten werden, gibt es weitere teilweise stark vereinfachte Modelle für
die Erstellung von Prototypen und zum Lehren. Außerdem bietet OpenSim die Möglichkeit von Benutzern
selbst erstellte Modelle aus einer Online Bibliothek zu laden.

Zum Analysieren und Simulieren von Modellen und Bewegungen gibt es Tools, um Markerdaten,
Gelenkkinematiken und externe Kräfte zu importieren und ebenfalls zu visualisieren. Einige Benutzer
haben Toolboxen erstellt und geteilt, die mit verschiedenen gängigen Bewegungserfassungssystemen
kompatibel sind.

\underline{Tools}\newline
Das Scale Tool erlaubt es Objektspezifisch Muskel Skelett Modelle basierend auf den experimentell
erfassten Daten zu erstellen.

Mit der statischen Optimierung kann das Muskel-Redundanzproblem gelöst werden, basierend auf
etablierten Verfahren aus der wissenschaftlichen Literatur.

Mit Hilfe des Computed Muscle Control (CMC)-Tool lassen sich muskelgetriebene Vorwärtssimulationen
erzeugen. Diese Methode wurde erfolgreich für verschiedene Bewegungen wie das Gehen, Laufen,
Radfahren, Springen und die Analyse pathologischer Gangmuster eingesetzt.

OpenSim ermöglicht außerdem eine detaillierte Analyse („Probing“) von Modellen und Simulationen.
Dabei können Größen wie Gelenkwinkel, Muskelkräfte, Muskelhebelarme, Muskelarbeit oder die Bewegung
des Körperschwerpunkts untersucht und grafisch dargestellt werden.

Zur Visualisierung bietet OpenSim eine grafische Benutzeroberfläche, in der nahezu alle
Modellkomponenten angezeigt werden können. Externe Geometrien (z. B. STL- oder OBJ-Dateien)
lassen sich importieren und mit integrierten Bild- und Video-Tools können anschauliche Darstellungen
für Präsentationen und Publikationen erstellt werden.
