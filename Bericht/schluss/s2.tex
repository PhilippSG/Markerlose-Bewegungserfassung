\section{Ausblick} %Mögliche Weiterentwicklungen und Verbesserungen
Das vorliegende Projekt stellt eine funktionale Pipeline zur markerlosen Bewegungserfassung und 
biomechanischen Auswertung in OpenSim bereit. 
Die erzielten Ergebnisse zeigen jedoch zugleich mehrere Ansatzpunkte für weiterführende Arbeiten, 
die von einer nachfolgenden Projektgruppe aufgegriffen und vertieft werden können.

Ein zentraler Schwerpunkt zukünftiger Arbeiten liegt in der weiteren Verbesserung und Erweiterung der bestehenden Pipeline. 
Insbesondere bietet sich ein systematischer Vergleich unterschiedlicher Pose-Estimation-Modelle an. 
Neben weiteren Modellen innerhalb von MediaPipe könnten auch alternative Frameworks wie OpenPose oder 
vergleichbare Open-Source-Ansätze untersucht werden, um deren Robustheit, Genauigkeit und Eignung für 
biomechanische Anwendungen zu bewerten.

Darüber hinaus kann die Genauigkeit der 3D-Rekonstruktion insbesondere im Pose2Sim-Ansatz weiter erhöht werden. 
Während im aktuellen Projekt ein Setup mit zwei Kameras betrachtet wurde, könnte der Einsatz zusätzlicher Kameras 
die Tiefenrekonstruktion stabilisieren und die Auswirkungen von Selbstüberdeckungen reduzieren. Eine Analyse des 
Einflusses der Kamerazahl, Positionierung und Kalibrierungsqualität auf die Rekonstruktionsergebnisse stellt dabei 
einen vielversprechenden Ansatz dar.

Ein weiterer möglicher Entwicklungsschritt betrifft die inverse Kinematik. 
Neben dem in OpenSim integrierten IK-Tool könnten alternative IK-Verfahren untersucht werden, 
beispielsweise durch eine eigene Implementierung oder durch den Vergleich mit anderen Optimierungsansätzen. 
Ziel wäre es, die Sensitivität gegenüber Markerfehlern zu reduzieren und stabilere Gelenkwinkelverläufe zu erzielen.

Auch die Skalierung der biomechanischen Modelle bietet weiteres Optimierungspotenzial. 
Insbesondere Körpersegmente mit variabler Länge, wie beispielsweise die Verbindung zwischen Schulter und Hüfte in markerlosen Pose-Schätzungen, 
stellen derzeit eine Herausforderung dar. Zukünftige Arbeiten könnten neue Skalierungsstrategien entwickeln, 
etwa durch zusätzliche geometrische Constraints, anthropometrische Modelle oder zeitliche Mittelung über geeignete Bewegungssequenzen.

Darüber hinaus könnte der Pose2Sim-Ansatz weiter verbessert und systematisch mit markerbasierten Bewegungserfassungssystemen verglichen werden. 
Da markerbasierte Systeme derzeit als Referenzstandard gelten, würde ein solcher Vergleich eine fundierte Validierung der markerlosen Pipeline ermöglichen und 
deren biomechanische Genauigkeit besser einordnen.

Ein weiterer vielversprechender Ansatz besteht in der Erweiterung der Pose-Estimation durch zusätzliche Sensorik. 
Beispielsweise könnte die Kombination von Smartphone-Kameras mit integrierten LiDAR-Sensoren genutzt werden, 
um zusätzliche Tiefeninformationen zu erfassen und die 3D-Rekonstruktion insbesondere bei monokularen Ansätzen zu verbessern. 
Die Fusion visueller und distanzbasierter Daten stellt dabei ein interessantes Forschungsfeld dar.

Insgesamt bietet das Projekt eine solide Grundlage für weiterführende Untersuchungen, 
die sowohl die methodische Genauigkeit als auch die praktische Anwendbarkeit markerloser Bewegungserfassungssysteme 
in biomechanischen Analysen weiter verbessern können.