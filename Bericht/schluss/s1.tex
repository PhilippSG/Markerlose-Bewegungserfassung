\chapter{Schluss}
\section{Fazit}
\subsection{Genauigkeit der 3D-Pose-Schätzung von MediaPipe}
Da dies ein monokularer Ansatz ist, sind die 3D Positionen von MediaPipe BlazePose
nicht so genau wie bei einem Mehrkamerasystem. Um die Genauigkeit zu überprüfen, haben wir ein
Beispielvideo verwendet
(\href{https://github.com/PhilippSG/Markerlose-Bewegungserfassung/blob/main/Videos/video_comp.mp4}{siehe GitHub-Repository, Videos\//video\_comp.mp4}). Im Video ist eine nahezu perfekte 2D Bewegung in der sichtbaren x-y-Ebene zu sehen, um die Stabilität der
3D Pose Schätzung zu testen. 

Zunächst haben wir das Video mit MediaPipe analysiert,
um die 3D Positionen der Keypoints zu extrahieren. Anschließend haben wir den Prozess wiederholt,
aber diesmal die Z-Koordinaten aller Keypoints auf Null gesetzt, um eine 2D-Analyse zu simulieren.
Danach haben wir die Ergebnisse in OpenSim importiert um sie zu betrachten.

Im Ergebnisvideo (\href{https://github.com/PhilippSG/Markerlose-Bewegungserfassung/blob/main/Bericht/vids/MP_Comp_2D-3D.mp4}{siehe GitHub-Repository, MediaPipe\//Bericht\//vids\//MP\_Comp\_2D-3D.mp4}),
so wie in der Abbildung~\ref{fig:2d-3d-vergleich}, ist deutlich zu erkennen, dass die 3D-Analyse von MediaPipe
zu erheblichen Schwankungen in der Z-Achse führt, obwohl die Bewegung eigentlich in der x-y-Ebene stattfindet.

\begin{figure}[ht]
    \centering
    \includegraphics[width=0.6\textwidth]{2D-3D_comp.png}
    \caption{Vergleich der 2D-(links) und 3D-(rechts) Analyse in OpenSim}
    \label{fig:2d-3d-vergleich} 
\end{figure}

Dies zeigt, dass die 3D Pose Schätzung von MediaPipe in diesem Fall nicht stabil genug ist, um präzise Bewegungsdaten zu liefern.
Im Gegensatz dazu liefert die 2D-Analyse, bei der die Z-Koordinaten auf Null gesetzt wurden, eine viel stabilere und genauere Darstellung der Bewegung.
Dies unterstreicht die Limitationen der 3D Pose Schätzung bei monokularen Systemen und legt nahe, dass für präzise biomechanische Analysen
Mehrkamerasysteme oder andere fortschrittlichere Methoden erforderlich sind.
