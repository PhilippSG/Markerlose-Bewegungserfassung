\chapter{Schluss}
\section{Fazit}
Im Rahmen dieses Projekts wurde untersucht, inwieweit markerlose Bewegungserfassungssysteme zur Rekonstruktion menschlicher Bewegungen 
und zur Bestimmung von Gelenkwinkeln in der biomechanischen Simulationsumgebung OpenSim eingesetzt werden können. 
Dabei wurden zwei unterschiedliche Ansätze betrachtet und miteinander verglichen: ein eigenentwickelter Workflow auf Basis eines monokularen
Pose-Schätzverfahrens sowie eine etablierte Multi-Kamera-Pipeline, die als Referenz diente.

Die Ergebnisse zeigen, dass markerlose Bewegungserfassung grundsätzlich geeignet ist,
kinematische Bewegungsdaten zu erfassen und diese erfolgreich in OpenSim zu integrieren. 
Insbesondere konnte demonstriert werden, dass aus videobasierten Poseschätzungen Markerdateien erzeugt werden können, 
die eine inverse Kinematik und damit die Berechnung von Gelenkwinkeln ermöglichen. Der entwickelte Workflow stellt dabei eine kostengünstige 
und flexibel einsetzbare Alternative zu klassischen markerbasierten Bewegungserfassungssystemen dar.

Gleichzeitig wurden jedoch auch deutliche Limitationen identifiziert. 
Insbesondere bei der monokularen 3D-Pose-Schätzung zeigten sich Instabilitäten, vor allem in der Tiefenrichtung, 
die sich in erhöhten Markerfehlern und starken Schwankungen einzelner Gelenkwinkel äußerten. Diese Effekte traten 
verstärkt bei Endlagen der Bewegung sowie bei Selbstüberdeckungen auf und schränken die biomechanische Aussagekraft der Ergebnisse ein. 
Der Vergleich mit der Multi-Kamera-Pipeline verdeutlichte, dass eine präzisere räumliche Rekonstruktion zu stabileren und 
physikalisch plausibleren Ergebnissen führt.

Darüber hinaus stellte sich die Skalierung der OpenSim-Modelle mit MediaPipe als kritischer Schritt heraus. 
Durch meherere Versuche mit Ganzkörpermodellen konnte gezeigt werden, dass die Genauigkeit der Gelenkwinkelberechnung
stark von der Anzahl der verwendeten Marker und der Qualität der Skalierung abhängt.
Modelle mit einer höheren Anzahl an Markern lieferten unseren Tests nach schlechtere Ergebnisse mit größeren Abweichungen,
da die Pose-Estimation in diesen Fällen nicht ausreichend genaue Landmarken lieferte und große Probleme bei der Tiefenerkennung auftraten.

Zusammenfassend zeigt das Projekt, dass markerlose Bewegungserfassung in Kombination mit biomechanischen Modellen 
ein hohes Potenzial für zukünftige Anwendungen besitzt, insbesondere dort, wo flexible, kostengünstige und alltagstaugliche Erfassungssysteme gefragt sind. 
Für präzise biomechanische Analysen sind jedoch weiterhin robuste 3D-Rekonstruktionsverfahren, eine sorgfältige Modellskalierung sowie eine Validierung der Ergebnisse erforderlich.
Außerdem wird man niemals eine genauso hohe Genauigkeit bei monokularen Systemen, wie im Vergleich zu Mehrkamerasystemen erreichen können,
da die Tiefeninformation bei Mehrkamerasystemen durch Triangulierung wesentlich besser bestimmt werden kann.