\chapter{Schluss}
\section{Fazit}
Im Rahmen dieses Projekts wurde untersucht, inwieweit markerlose Bewegungserfassungssysteme zur Rekonstruktion menschlicher Bewegungen 
und zur Bestimmung von Gelenkwinkeln in der biomechanischen Simulationsumgebung OpenSim eingesetzt werden können. 
Dabei wurden zwei unterschiedliche Ansätze betrachtet und miteinander verglichen: ein eigenentwickelter Workflow auf Basis eines monokularen
Pose-Schätzverfahrens sowie eine etablierte Multi-Kamera-Pipeline, die als Referenz diente.

Die Ergebnisse zeigen, dass markerlose Bewegungserfassung grundsätzlich geeignet ist,
kinematische Bewegungsdaten zu erfassen und diese erfolgreich in OpenSim zu integrieren. 
Insbesondere konnte demonstriert werden, dass aus videobasierten Poseschätzungen Markerdateien erzeugt werden können, 
die eine inverse Kinematik und damit die Berechnung von Gelenkwinkeln ermöglichen. Der entwickelte Workflow stellt dabei eine kostengünstige 
und flexibel einsetzbare Alternative zu klassischen markerbasierten Bewegungserfassungssystemen dar.


\textbf{Gegenüberstellung der Vor- und Nachteile}\newline
\underline{Genauigkeit und biomechanische Qualität}\newline
Im Zuge der Untersuchung konnten deutliche Unterschiede in der Rekonstruktionsqualität der beiden Ansätze festgestellt. 
Der monokulare MediaPipe-Ansatz weist insbesondere in der 3D-Pose-Schätzung Limitationen auf, die sich vor allem in Instabilitäten der Tiefenrichtung äußern. 
Diese führen zu erhöhten Markerfehlern und starken Schwankungen einzelner Gelenkwinkel, insbesondere in den Endlagen der Bewegung, bei Selbstüberdeckungen sowie bei geneigter Kameraperspektive. 
Da MediaPipe ohne explizite geometrische Kamerakalibrierung arbeitet, ist die räumliche Rekonstruktion empfindlich und fehleranfälliger gegenüber Perspektivänderungen.

Der Vergleich mit der Pose2Sim-OpenSim-Pipeline zeigt hingegen, dass die multikamerabasierte Triangulation mit geometrisch kalibrierten Kameras 
stabilere und biomechanisch konsistentere 3D-Rekonstruktionen ermöglicht. Pose2Sim erweist sich damit als deutlich präziser für biomechanische Analysen.

Zusätzlich stellte sich die Skalierung der OpenSim-Modelle bei MediaPipe als kritischer Faktor heraus. Die Genauigkeit der Gelenkwinkelberechnung hängt 
stark von der Anzahl und Qualität der verwendeten Marker ab. Insbesondere bei Ganzkörpermodellen führten unzureichend präzise Landmarken und Probleme in der 
Tiefenschätzung zu größeren Abweichungen, wodurch die biomechanische Aussagekraft weiter eingeschränkt wird.

\underline{Technischer Aufwand/Komplexität und Benutzerfreundlichkeit}\newline
Im Hinblick auf den technischen Aufwand und Benutzerfreundlichkeit zeigen sich folgende Vor-und Nachteile beider Ansätze.

Die Pose2Sim-OpenSim-Pipeline erfordert ein komplexes Setup mit zwei (oder mehr) Kameras, präziser Kalibrierung und 
einer mehrstufigen Verarbeitungskette. Initial ist die Einrichtung aufwendiger, jedoch steht anschließend eine weitgehend 
standardisierte und reproduzierbare Verarbeitungskette zur Verfügung.

Bei MediaPipe ist die initiale Erstellung einer biomechanisch verwertbaren Pipeline mit erheblichem Entwicklungsaufwand verbunden, 
insbesondere für die Skalierung der Daten, die Transformation der Koordinatensysteme und die Integration in OpenSim. Ist diese Pipeline 
jedoch einmal implementiert, gestaltet sich die anschließende Anwendung vergleichsweise benutzerfreundlich und unkompliziert, da vor allem 
kein aufwendiges Kamerasetup mit geometrischer Kalibrierung erforderlich ist.

\underline{Alltagstauglichkeit}\newline
MediaPipe erweist sich als alltagstauglicher, insbesondere für mobile Einsatzszenarien, da es mit minimaler Hardware ohne aufwendiges 
Kamerasetup flexibel in unterschiedlichen Umgebungen eingesetzt werden kann. Der Pose2Sim-Ansatz erfordert eine präzise Kamerakalibrierung 
und ist dadurch in seiner Einsatzfähigkeit eher unflexibel.

\underline{Kosten und Skalierbarkeit}\newline
Auch hinsichtlich der Kosten unterscheiden sich die beiden Ansätze. Der MediaPipe-Workflow kann mit handelsüblicher Hardware betrieben werden 
und erfordert in der Regel lediglich eine Standardkamera sowie einen Rechner. Dadurch ist der Ansatz kostengünstig skalierbar und auch für den 
Einsatz außerhalb spezialisierter Labore geeignet.

Die Pose2Sim-OpenSim-Pipeline ist demgegenüber mit höheren Investitionskosten verbunden. Für den Betrieb sind zwei oder mehr Kameras, sowie eine 
(leistungsstarke) Grafikkarte erforderlich, um die umfangreichen Bilddaten verarbeiten zu können.

\textbf{Geeignete Anwendungsgebiete von MediaPipe und Pose2Sim}\newline
Aus den Ergebnissen dieser Arbeit lassen sich Anwendungsfelder für die beiden untersuchten Ansätze ableiten. MediaPipe eignet sich insbesondere für mobile, 
flexible und alltagsnahe Anwendungen, bei denen eine schnelle und unkomplizierte Bewegungserfassung im Vordergrund steht und keine hochpräzise biomechanische 
Analyse zwingend erforderlich ist. Mögliche Einsatzgebiete wären Fitness- und Trainings-Apps, Bewegungstracking im Alltag, Gaming, einfache Haltungsanalysen sowie Voranalysen.

Die Pose2Sim-OpenSim-Pipeline ist hingegen vor allem für wissenschaftliche und klinische Anwendungen geeignet, bei denen eine hohe Genauigkeit und biomechanische Konsistenz
der Bewegungsdaten erforderlich sind. Besonders vorteilhaft ist der Einsatz in Umgebungen, in denen das Kamerasetup über längere Zeit unverändert bleibt, beispielsweise in Laboren, 
beispielsweise bei Entwicklung humanoider Roboter oder Kliniken, z.B. für präzise Ganganalysen, Rehabilitation nach Verletzungen,etc. In solchen Szenarien kann die aufwendige Kalibrierung 
langfristig genutzt werden und die hohe Rekonstruktionsqualität optimal genutzt werden.

\textbf{Einordnung der Ergebnisse}\newline
Zusammenfassend zeigt das Projekt, dass markerlose Bewegungserfassung in Kombination mit biomechanischen Modellen ein hohes Potenzial für zukünftige Anwendungen besitzt, insbesondere dort, 
wo flexible, kostengünstige und alltagstaugliche Erfassungssysteme gefragt sind.
 
Für präzise biomechanische Analysen sind jedoch weiterhin robuste 3D-Rekonstruktionsverfahren, eine sorgfältige Modellskalierung sowie eine Validierung der Ergebnisse erforderlich. 
Außerdem wird man niemals eine genauso hohe Genauigkeit bei monokularen Systemen, wie im Vergleich zu Mehrkamerasystemen erreichen können, da die Tiefeninformation bei Mehrkamerasystemen 
durch Triangulierung wesentlich besser bestimmt werden kann.


% Gleichzeitig wurden jedoch auch deutliche Limitationen identifiziert. 
% Insbesondere bei der monokularen 3D-Pose-Schätzung zeigten sich Instabilitäten, vor allem in der Tiefenrichtung, 
% die sich in erhöhten Markerfehlern und starken Schwankungen einzelner Gelenkwinkel äußerten. Diese Effekte traten 
% verstärkt bei Endlagen der Bewegung sowie bei Selbstüberdeckungen auf und schränken die biomechanische Aussagekraft der Ergebnisse ein. 
% Der Vergleich mit der Multi-Kamera-Pipeline verdeutlichte, dass eine präzisere räumliche Rekonstruktion zu stabileren und 
% physikalisch plausibleren Ergebnissen führt.

% Darüber hinaus stellte sich die Skalierung der OpenSim-Modelle mit MediaPipe als kritischer Schritt heraus. 
% Durch meherere Versuche mit Ganzkörpermodellen konnte gezeigt werden, dass die Genauigkeit der Gelenkwinkelberechnung
% stark von der Anzahl der verwendeten Marker und der Qualität der Skalierung abhängt.
% Modelle mit einer höheren Anzahl an Markern lieferten unseren Tests nach schlechtere Ergebnisse mit größeren Abweichungen,
% da die Pose-Estimation in diesen Fällen nicht ausreichend genaue Landmarken lieferte und große Probleme bei der Tiefenerkennung auftraten.
